\chapter{Viertes Kapitel}
\section{Listen und Aufz\"ahlungen}

Hier mal eine Auflistung von Elementen
\begin{itemize}
 \item erstes Element
 \item zweites Element
 \item noch ein Element
\end{itemize}

Hier mal eine Aufz\"ahlung
\begin{enumerate}
 \item erster Punkt
 \item noch ein Punkt
 \item letzter Punkt
\end{enumerate}

% Neue Seite
\newpage

\section{Und n\"achster Abschnitt etwas l\"anger als vorher es war}
Eine neue Seite, um auchmal die Kopfzeile zu sehen, da sie auf Seiten mit Kapitelanfang nicht erscheinen. Eine Abk\"urzung ist z.B. etc.\abk{etc.}{et cetera}.

% Neue Seite
\newpage
\section{Eine Tabelle}

Hier eine Tabelle:
\begin{table}[htbp]
\centering
\begin{tabular}{l|l|l|l}
SpalteA & SpalteB & SpalteC & SpalteD \\
\midrule
InhaltA1 & InhaltB1 & InhaltC1 & InhaltD1 \\
InhaltA2 & InhaltB2 & InhaltC2 & InhaltD2 \\
InhaltA3 & InhaltB3 & InhaltC3 & InhaltD3
\end{tabular}
\caption{Beispiel einer Tabelle}
\label{tab:tabelle1}
\end{table}

Wie man in der Tabelle~\ref{tab:tabelle1} sehen kann ...
